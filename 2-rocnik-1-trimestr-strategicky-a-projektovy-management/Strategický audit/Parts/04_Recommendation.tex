\newcommand\StrategickeRecommendations{\texorpdfstring{\MakeUppercase{\romannumeral 4}.}{} FORMULACE DOPORUČENÍ Z AUDITU STRATEGIE}
\begin{lefttextpipe}
	{\Large \StrategickeRecommendations}
\end{lefttextpipe}
\phantomsection
\addcontentsline{toc}{section}{\StrategickeRecommendations}

%TODO: Projít celý text a podívat se, kde říkám, že udělám doporučení v projektové části, ty pak shrnout sem.

Společnost je obecně dle analýzy v dobrém stavu, má dlouhodobě kladné EVA a zaznaménává dlouhodobý růst s menším WACC. Společnost je dále stabilní i co se týče interních faktorů a většiny externích faktorů, společnost však nevykonává činnost, která by byla z hlediska konkurence jedinečná a nenapodobitelná, potenciální konkurence tak pro ní představuje potenciálně zásadní problém.\\

I díky tomu se společnost zařadila do skupiny HLB poradenských firem, v rámci které se může zlepšovat, zefektivňovat své procesy a obecně přebírat pozitivní změny, jedná se tak o faktickou kooperativní strategii, i když nemusí nutně být zaměřena na finanční faktory.\\

V rámci českého trhu by si měla společnost udržet hlavně dobré jméno, pověst a kvalitu (jedná se v podstatě o jediný způsob, krom cenotvorby, jak mohou nastavit konkurenční strategii oproti ostatním subjektům poskytujícím podobné či úplně stejné služby), neboť český trh je poradenskými společnostmi saturován, a každý drobný přešlap by tak mohl mít na analyzovanou společnost katastrofální důsledky.\\

Společnost se však celkem nachází ve velice příznivém postavení a autoři v kontextu celé analýzy tak nemohou doporučit zásadnější změny. Dva body, které přestou autoři doporučí je zásadnější změna marketingové strategie, v průběhu celého procesu shromažďování informací autoři shledali, že ve veřejném prostoru absolutně absentuje jakákoliv propagační činnost společnosti. Jako druhý bod pak autoři načrtávají návrh na implementaci Green dealu (viz kapitola \textit{V. Projektová část}.\\

%Co se týká ostatních bodů doporučení v oblasti plánování výzkumné činnosti, informatické činnosti, výrobní (cyklus) a procesních a personálních plánů.

\noindent\textbf{Ostatní body doporučení}\\

Plánování výzkumné činnosti není pro společnost zásadní, nepotřebuje postupovat zásadnější těchnologické inovace, výzkumnou činnost je tak možné vztáhnout na čistě vědecký a procesní výzkum stavu legislativy v oblasti účetní, daňové a auditorské.\\

ICT procesy nepotřebují zásadní revizi, společnost v rámci technologické sféry pracuje správně.\\

Hodnotových řetězců firma poskytuje všechny oblasti své činnosti, není tedy třeba doporučovat zásadní změnu, vždy je však možná expanze ve smyslu poskytování nových služeb (viz kapitola \textit{A. Hodnotový řetězec).}\\

Procesně a personálně je společnost rovněž velmi zralá, autoři se domnívají, že by stálo za zvážení implementace holistického systému v rámci organizační struktury analyzovaného podniku pro seniornější zaměstnance, ačkoliv je struktura již teď velice plochá, tím by se potenciálně urychlilo zapojení nových kolegů (hlavně vzhledem k zákoným podmínkácm kladeným na ně zákonodárcem) do procesů společnosti a všichni zaměstnanci by se blíže mohli podílet na fungování společnosti.\\


%TODO: Najít kde mám použito holistický a nahradit to jiným slovem

%najít jakékoliv informace o společnosti, které by měli alespoň potenciální podobu marketingových / propagačních  

Tady něco napíšu.

%\rotatebox{270}{\rlap{\hbox{Krátký vliv}}}

\begin{table}[]
\centering
\makebox[\textwidth]{
\resizebox{20cm}{!}{%
\begin{tabular}{|l|l|l|l|l|llll|}
\hline
 & Strategické faktory & Váha (\%) & Rating (1 - 5) & Vážené skóre & \multicolumn{1}{l|}{\rotatebox{270}{\rlap{\hbox{Krátký vliv}}}} & \multicolumn{1}{l|}{\rotatebox{270}{\rlap{\hbox{Střední (1 - 3)}}}} & \multicolumn{1}{l|}{\rotatebox{270}{\rlap{\hbox{Dlouhodobý}}}} & Komentáře, odůvodnění, odkazy \\[70pt] \hline
1 & IF: Značka & 12\% & 4 & 0,48 & \multicolumn{1}{l|}{} & \multicolumn{1}{l|}{} & \multicolumn{1}{l|}{X} & Společnost si zakládá \\
 &  &  &  &  & \multicolumn{1}{l|}{} & \multicolumn{1}{l|}{} & \multicolumn{1}{l|}{} & na značce a dobrém jméně, to se jí daří udržet. \\ \hline
2 & IF: Kvalita & 15\% & 4 & 0,60 & \multicolumn{1}{l|}{} & \multicolumn{1}{l|}{} & \multicolumn{1}{l|}{X} & Společnost poskytuje dlouhodobě \\
 &  &  &  &  & \multicolumn{1}{l|}{} & \multicolumn{1}{l|}{} & \multicolumn{1}{l|}{} & kvalitní služby, tato úroveň kvality \\
 &  &  &  &  & \multicolumn{1}{l|}{} & \multicolumn{1}{l|}{} & \multicolumn{1}{l|}{} & se jí zatím daří udržet, nejsou s ní spojováný žádné problémy. \\ \hline
3 & IF: Know-how & 13\% & 5 & 0,65 & \multicolumn{1}{l|}{} & \multicolumn{1}{l|}{X} & \multicolumn{1}{l|}{} & Společnosti se daří dobře držet know-how \\
 &  &  &  &  & \multicolumn{1}{l|}{} & \multicolumn{1}{l|}{} & \multicolumn{1}{l|}{} & na uspokojivé až výborné úrovni. \\ \hline
4 & IF: Marketing & 10\% & 2 & 0,20 & \multicolumn{1}{l|}{X} & \multicolumn{1}{l|}{} & \multicolumn{1}{l|}{} & Společnost se na marketing v postatě nezaměřuje, doporučení je, \\
 &  &  &  &  & \multicolumn{1}{l|}{} & \multicolumn{1}{l|}{} & \multicolumn{1}{l|}{} & aby se začala více propagovat například i ve vztahu k obratově menším \\
 &  &  &  &  & \multicolumn{1}{l|}{} & \multicolumn{1}{l|}{} & \multicolumn{1}{l|}{} & subjektům a posílila tak svojí konkurenční výhodu na trhu. \\ \hline
5 & EF: Konkurence & 10\% & 3 & 0,30 & \multicolumn{1}{l|}{} & \multicolumn{1}{l|}{X} & \multicolumn{1}{l|}{} & Společnost v tuto chvíli není zásadněji připravena \\
 &  &  &  &  & \multicolumn{1}{l|}{} & \multicolumn{1}{l|}{} & \multicolumn{1}{l|}{} & na konkurenci na trhu, viz předchozí bod. \\ \hline
6 & EF: Růst ekonomiky a inflace & 10\% & 4 & 0,40 & \multicolumn{1}{l|}{} & \multicolumn{1}{l|}{} & \multicolumn{1}{l|}{X} & Společnost je dle finanční analýzy dobře připravena na výkyvy na trhu. \\ \hline
7 & EF: Změna legislativy & 15\% & 5 & 0,75 & \multicolumn{1}{l|}{X} & \multicolumn{1}{l|}{} & \multicolumn{1}{l|}{} & Společnosti se daří dobře reagovat na změny legislativy. \\ \hline
8 & EF: Poptávka & 15\% & 4 & 0,60 & \multicolumn{1}{l|}{} & \multicolumn{1}{l|}{X} & \multicolumn{1}{l|}{} & Společnost je dle finanční analýzy \\
 &  &  &  &  & \multicolumn{1}{l|}{} & \multicolumn{1}{l|}{} & \multicolumn{1}{l|}{} & ve středním období dobře připravena na výkyvy poptávky. \\ \hline
\cellcolor[HTML]{C0C0C0} & \cellcolor[HTML]{C0C0C0} & 100\% & \cellcolor[HTML]{C0C0C0} & Suma = 3,98 & \multicolumn{4}{l|}{Podnik je orientačně dobře strategicky připraven dosahuje středního skóre hlavně kvůli horšímu marketingu} \\
\cellcolor[HTML]{C0C0C0} & \cellcolor[HTML]{C0C0C0} &  & \cellcolor[HTML]{C0C0C0} &  & \multicolumn{4}{l|}{a horší připravenosti na konkurenci skóre mezi 3,5 až 4,5.} \\ \hline
\end{tabular}
}
}
\end{table}