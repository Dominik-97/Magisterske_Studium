\newcommand\SkenovaniVnitrnihoProstrediFirmy{\texorpdfstring{\MakeUppercase{\romannumeral 3}.}{} SKENOVÁNÍ VNĚITŘNÍHO PROSTŘEDÍ FIRMY}
\begin{lefttextpipe}
	{\Large \SkenovaniVnitrnihoProstrediFirmy}
\end{lefttextpipe}
\phantomsection
\addcontentsline{toc}{section}{\SkenovaniVnitrnihoProstrediFirmy}

\section*{A. Hodnotový řetězec}
\label{sec:Hodnotovy retezec}
\addcontentsline{toc}{subsection}{\nameref{sec:Hodnotovy retezec}}

Co se týče hodnotového řetězce, není u společnosti, která psokytuje služby možné nad ním přemýšlet jako na klasickým dodavatelským, výrobním a odběratelským řetězcem. To neznamená, že u u \textit{service industry} není možné definovat hodnotový řetězec\footfullcite{noauthor_supply_nodate}, jen je se nad ním potřeba zamyslet v jiném než tradičním smyslu.\\

Lze ho téměř pojmout v klasickém procesním (BPMN) smyslu.

Přijde klient (Akvizice) -> Předá informace (Dodávka) -> Informace se zpracují, vyhodnotí a vytvoří se očekávaný delivery (virutální) (výroba) -> Následně jsou informace předány zamýšlenému receivee (dodání).

Jedná se téměř v podstatě o popis procesu.

Sem dát ten diagram.

\begin{figure}[!hbtp]
	\centering
	\includegraphics[width=0.8\linewidth]{Parts/InternalResources/DIAGRAM_dodavatelsky_retezec.pdf}
	\caption[Dodavatelský řetězec]{Dodavatelský řetězec\footnote{Vlastní zpracování dle Griffin \& Co. Strategic Marketing Methods.}}
	\label{fig:Dodavatelsky retezec}
\end{figure}

V rámci tohoto industry to nelze pojmout jakkoli jinak - mají to stejně, jen si do toho diagramu musíme pluginnout daňové a účetní služby, klienti jsou většinou podnikatelé a odběratelé jsou často úřady.

Nikam se moc nemá šanci posunout v řetězci jako takovém, protože tento je takový a jiný nebude a oni pokrývají veškeré jeho části, jediná možnost jak expandovat je, není pokrýt jeho větší část, protože to už mají pokryté celé, ale vyvořit v podstatě nový, v rámci kterého budou nabízet nové služby.

Tedy rozšiřování by expansion of services. 

\newpage

\section*{B. Životní cyklus produktů (volitelně BCG matice)}
\label{sec:Zivotni cyklus produktu}
\addcontentsline{toc}{subsection}{\nameref{sec:Zivotni cyklus produktu}}

Neznáme interní tržby za jednotlivé oblasti, pouze aggregovanou hodnout publikovanou v rozvaze. Bude jenom odhadní, popsat jednotlivé oblasti.

Produkty: daně, účetnictví, audit

\begin{table}[!hbtp]
\centering
\begin{tabular}{ll|cc|}
\cline{3-4}
 &  & \multicolumn{2}{c|}{Relativní tržní podíl v procentech} \\ \cline{3-4} 
 &  & \multicolumn{1}{c|}{Vysoký} & Nízký \\ \hline
\multicolumn{1}{|c|}{} & \multicolumn{1}{c|}{Vysoký} & \multicolumn{1}{c|}{\cellcolor[HTML]{67FD9A}Sup1} & - Daně \newline - účetnictví \newline - audit \\[50pt] \cline{2-4} 
\multicolumn{1}{|c|}{\multirow{-2}{*}{Růst trhu v procentech}} & \multicolumn{1}{c|}{Nízký} & \multicolumn{1}{c|}{Sup3} & \cellcolor[HTML]{FD6864}{\color[HTML]{000000} Sup4} \\[50pt] \hline
\end{tabular}
\end{table}

\begin{table}[!hbtp]
\centering
\begin{tabular}{cc|cc|}
\cline{3-4}
\multicolumn{1}{l}{} & \multicolumn{1}{l|}{} & \multicolumn{2}{c|}{Relativní tržní podíl v procentech} \\ \cline{3-4} 
\multicolumn{1}{l}{} & \multicolumn{1}{l|}{} & \multicolumn{1}{c|}{Vysoký} & Nízký \\ \hline
\multicolumn{1}{|c|}{} &  & \multicolumn{1}{c|}{\cellcolor[HTML]{67FD9A}} & - a \\
\multicolumn{1}{|c|}{} & Vysoký & \multicolumn{1}{c|}{\cellcolor[HTML]{67FD9A}} & - b \\
\multicolumn{1}{|c|}{Růst trhu v} &  & \multicolumn{1}{c|}{\cellcolor[HTML]{67FD9A}} & - c \\ \cline{2-4} 
\multicolumn{1}{|c|}{procentech} &  & \multicolumn{1}{c|}{} & \cellcolor[HTML]{FD6864}{\color[HTML]{000000} } \\
\multicolumn{1}{|c|}{} & Nízký & \multicolumn{1}{c|}{} & \cellcolor[HTML]{FD6864} \\
\multicolumn{1}{|c|}{} &  & \multicolumn{1}{c|}{} & \cellcolor[HTML]{FD6864} \\ \hline
\end{tabular}
\end{table}

\section*{C. Životní cyklus firmy}
\label{sec:Zivotni cyklus firmy}
\addcontentsline{toc}{subsection}{\nameref{sec:Zivotni cyklus firmy}}

\section*{D. Organizační struktura a kultura podniku}
\label{sec:Organizacni struktura}
\addcontentsline{toc}{subsection}{\nameref{sec:Organizacni struktura}}

\inputfile{Parts/InternalResources/DIAGRAM_Organizational_structure}

Seniorové, manažeři se mohou prolínat.

\section*{E. Kapitálová struktura podniku}
\label{sec:Kapitalova struktura podniku}
\addcontentsline{toc}{subsection}{\nameref{sec:Kapitalova struktura podniku}}

\section*{F. Akademické analýzy}
\label{sec:Akademicke analyzy}
\addcontentsline{toc}{subsection}{\nameref{sec:Akademicke analyzy}}

\section*{G. V-R-I-O}
\label{sec:V-R-I-O}
\addcontentsline{toc}{subsection}{\nameref{sec:V-R-I-O}}