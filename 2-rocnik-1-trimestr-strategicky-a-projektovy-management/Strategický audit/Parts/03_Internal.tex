\newcommand\SkenovaniVnitrnihoProstrediFirmy{\texorpdfstring{\MakeUppercase{\romannumeral 3}.}{} SKENOVÁNÍ VNĚITŘNÍHO PROSTŘEDÍ FIRMY}
\begin{lefttextpipe}
	{\Large \SkenovaniVnitrnihoProstrediFirmy}
\end{lefttextpipe}
\phantomsection
\addcontentsline{toc}{section}{\SkenovaniVnitrnihoProstrediFirmy}

\section*{A. Hodnotový řetězec}
\label{sec:Hodnotovy retezec}
\addcontentsline{toc}{subsection}{\nameref{sec:Hodnotovy retezec}}

Co se týče hodnotového řetězce, není u společnosti, která psokytuje služby možné nad ním přemýšlet jako na klasickým dodavatelským, výrobním a odběratelským řetězcem. To neznamená, že u u \textit{service industry} není možné definovat hodnotový řetězec\footfullcite{noauthor_supply_nodate}, jen je se nad ním potřeba zamyslet v jiném než tradičním smyslu.\\

Lze ho téměř pojmout v klasickém procesním (BPMN) smyslu.

Přijde klient (Akvizice) -> Předá informace (Dodávka) -> Informace se zpracují, vyhodnotí a vytvoří se očekávaný delivery (virutální) (výroba) -> Následně jsou informace předány zamýšlenému receivee (dodání).

Jedná se téměř v podstatě o popis procesu.

Sem dát ten diagram.

\begin{figure}[!hbtp]
	\centering
	\includegraphics[width=0.8\linewidth]{Parts/InternalResources/DIAGRAM_dodavatelsky_retezec.pdf}
	\caption[Dodavatelský řetězec]{Dodavatelský řetězec\footnote{Vlastní zpracování dle Griffin \& Co. Strategic Marketing Methods.}}
	\label{fig:Dodavatelsky retezec}
\end{figure}

V rámci tohoto industry to nelze pojmout jakkoli jinak - mají to stejně, jen si do toho diagramu musíme pluginnout daňové a účetní služby, klienti jsou většinou podnikatelé a odběratelé jsou často úřady.

Nikam se moc nemá šanci posunout v řetězci jako takovém, protože tento je takový a jiný nebude a oni pokrývají veškeré jeho části, jediná možnost jak expandovat je, není pokrýt jeho větší část, protože to už mají pokryté celé, ale vyvořit v podstatě nový, v rámci kterého budou nabízet nové služby.

Tedy rozšiřování by expansion of services. 

\newpage

\section*{B. Životní cyklus produktů (volitelně BCG matice)}
\label{sec:Zivotni cyklus produktu}
\addcontentsline{toc}{subsection}{\nameref{sec:Zivotni cyklus produktu}}

Neznáme interní tržby za jednotlivé oblasti, pouze aggregovanou hodnout publikovanou v rozvaze. Bude jenom odhadní, popsat jednotlivé oblasti.

Produkty: daně, účetnictví, audit

\begin{table}[!hbtp]
\centering
\begin{tabular}{cc|cc|}
\cline{3-4}
\multicolumn{1}{l}{} & \multicolumn{1}{l|}{} & \multicolumn{2}{c|}{Relativní tržní podíl v procentech} \\ \cline{3-4} 
\multicolumn{1}{l}{} & \multicolumn{1}{l|}{} & \multicolumn{1}{c|}{Vysoký} & Nízký \\ \hline
\multicolumn{1}{|c|}{} &  & \multicolumn{1}{c|}{\cellcolor[HTML]{67FD9A}} & - daně \\
\multicolumn{1}{|c|}{} & Vysoký & \multicolumn{1}{c|}{\cellcolor[HTML]{67FD9A}} & - účetnictví \\
\multicolumn{1}{|c|}{Růst trhu v} &  & \multicolumn{1}{c|}{\cellcolor[HTML]{67FD9A}} & - audit \\ \cline{2-4} 
\multicolumn{1}{|c|}{procentech} &  & \multicolumn{1}{c|}{} & \cellcolor[HTML]{FD6864}{\color[HTML]{000000} } \\
\multicolumn{1}{|c|}{} & Nízký & \multicolumn{1}{c|}{} & \cellcolor[HTML]{FD6864} \\
\multicolumn{1}{|c|}{} &  & \multicolumn{1}{c|}{} & \cellcolor[HTML]{FD6864} \\ \hline
\end{tabular}
\end{table}

\section*{C. Životní cyklus firmy}
\label{sec:Zivotni cyklus firmy}
\addcontentsline{toc}{subsection}{\nameref{sec:Zivotni cyklus firmy}}

\begin{table}[]
\centering
\makebox[\textwidth]{
\begin{tabular}{l|c|c|c|c|c|c|c|c|c|c|}
\cline{2-11}
 & 2010 & 2011 & 2012 & 2013 & 2014 & 2015 & 2016 & 2017 & 2018 & 2019 \\ \hline
\multicolumn{1}{|l|}{Obrat} & 67 974 & 64 259 & 61 174 & 63 775 & 63 594 & 67 197 & 72 345 & 76 690 & 92 918 & 104 449 \\ \hline
\multicolumn{1}{|l|}{Růst obratu} & 6 814 & -3 715 & -3 085 & 2 601 & -181 & 3 603 & 5 148 & 4 345 & 16 228 & 11 531 \\ \hline
\rowcolor[HTML]{C0C0C0} 
\multicolumn{1}{|l|}{\cellcolor[HTML]{C0C0C0}Procentuální růst obratu} & 11,1413\% & -5,4653\% & -4,8009\% & 4,2518\% & -0,2838\% & 5,6656\% & 7,6611\% & 6,0059\% & 21,1605\% & 12,4099\% \\ \hline
\multicolumn{1}{|l|}{Čistý zisk} & 11 690 & 13 399 & 9092 & 12 622 & 10 314 & 11 054 & 12 619 & 13 181 & 18 063 & 22 571 \\ \hline
\multicolumn{1}{|l|}{Růst čistého zisku} & 4 728 & 1 709 & -4 307 & 3 530 & -2 308 & 740 & 1 565 & 562 & 4 882 & 4 508 \\ \hline
\rowcolor[HTML]{C0C0C0} 
\multicolumn{1}{|l|}{\cellcolor[HTML]{C0C0C0}Procentuální růst čistého   zisku} & 67,9115\% & 14,6193\% & -32,1442\% & 38,8253\% & -18,2855\% & 7,1747\% & 14,1578\% & 4,4536\% & 37,0382\% & 24,9571\% \\ \hline
\multicolumn{1}{|l|}{Bilanční suma} & 38581 & 30801 & 27022 & 30758 & 29937 & 31164 & 36115 & 33632 & 36739 & 44179 \\ \hline
\multicolumn{1}{|l|}{Růst bilanční sumy} & 6506 & -7780 & -3779 & 3736 & -821 & 1227 & 4951 & -2483 & 3107 & 7440 \\ \hline
\rowcolor[HTML]{C0C0C0} 
\multicolumn{1}{|l|}{\cellcolor[HTML]{C0C0C0}Procentuální růst bilanční   sumy} & 20,2837\% & -20,1654\% & -12,2691\% & 13,8258\% & -2,6692\% & 4,0986\% & 15,8869\% & -6,8753\% & 9,2382\% & 20,2510\% \\ \hline
\multicolumn{1}{|l|}{Růst HDP} & 0,8156\% & 1,7461\% & 0,7325\% & 1,3207\% & 4,8969\% & 6,5173\% & 3,6092\% & 6,7425\% & 5,8273\% & 6,9605\% \\ \hline
\end{tabular}}
\end{table}

\section*{D. Organizační struktura a kultura podniku}
\label{sec:Organizacni struktura}
\addcontentsline{toc}{subsection}{\nameref{sec:Organizacni struktura}}

\inputfile{Parts/InternalResources/DIAGRAM_Organizational_structure}

Seniorové, manažeři se mohou prolínat.

\section*{E. Kapitálová struktura podniku}
\label{sec:Kapitalova struktura podniku}
\addcontentsline{toc}{subsection}{\nameref{sec:Kapitalova struktura podniku}}

\section*{F. Akademické analýzy}
\label{sec:Akademicke analyzy}
\addcontentsline{toc}{subsection}{\nameref{sec:Akademicke analyzy}}

\section*{G. V-R-I-O}
\label{sec:V-R-I-O}
\addcontentsline{toc}{subsection}{\nameref{sec:V-R-I-O}}