\newcommand\SkenovaniVnitrnihoProstrediFirmy{\texorpdfstring{\MakeUppercase{\romannumeral 3}.}{} SKENOVÁNÍ VNITŘNÍHO PROSTŘEDÍ FIRMY}
\begin{lefttextpipe}
	{\Large \SkenovaniVnitrnihoProstrediFirmy}
\end{lefttextpipe}
\phantomsection
\addcontentsline{toc}{section}{\SkenovaniVnitrnihoProstrediFirmy}

\section*{A. Hodnotový řetězec}
\label{sec:Hodnotovy retezec}
\addcontentsline{toc}{subsection}{\nameref{sec:Hodnotovy retezec}}

Co se týče hodnotového řetězce, není u společnosti, která psokytuje služby možné nad ním přemýšlet jako na klasickým dodavatelským, výrobním a odběratelským řetězcem. To neznamená, že u \textit{service industry} není možné definovat hodnotový řetězec\footfullcite{noauthor_supply_nodate}, jen je se nad ním potřeba zamyslet v jiném než tradičním smyslu.\\

Lze ho téměř pojmout v klasickém procesním (BPMN) smyslu.\\

%Přijde klient (Akvizice) -> Předá informace (Dodávka) -> Informace se zpracují, vyhodnotí a vytvoří se očekávaný delivery (virutální) (výroba) -> Následně jsou informace předány zamýšlenému receivee (dodání).

%Jedná se téměř v podstatě o popis procesu.

%Sem dát ten diagram.

Diagram hodnotového řetězce pro společnost poskytující služby by se dal vyobrazit následovně:\\

\begin{figure}[!hbtp]
	\centering
	\includegraphics[width=0.8\linewidth]{Parts/InternalResources/DIAGRAM_dodavatelsky_retezec.pdf}
	\caption[Dodavatelský řetězec]{Dodavatelský řetězec\footnote{Vlastní zpracování dle Griffin \& Co. Strategic Marketing Methods.}}
	\label{fig:Dodavatelsky retezec}
\end{figure}

V rámci \textit{service industry} nelze hodnotový řetězec pojmout zásadněji v jiné podobě, jednotlivé společnosti poskytující služby mají svojí strukturou podobné hodnotové řetězce, liší se jenom v poskytovaných službách. S ohledem na toto je potřeba na diagram hledět v kontextu daňové a účetní společnosti. Klienti jsou standardně subjekty podnikající, odběratelé mohou být v mnoha případech úřady.\\

% jen si do toho diagramu musíme pluginnout daňové a účetní služby, klienti jsou většinou podnikatelé a odběratelé jsou často úřady.

%Nikam se moc nemá šanci posunout v řetězci jako takovém, protože tento je takový a jiný nebude a oni pokrývají veškeré jeho části, jediná možnost jak expandovat je, není pokrýt jeho větší část, protože to už mají pokryté celé, ale vyvořit v podstatě nový, v rámci kterého budou nabízet nové služby.

Společnost s ohledem na výše popsané nemá zásadnější možnosti posunout se v hodnotovém řetězci, popřípadě zaměřit se na expanzi v rámci jeho jednotlivých částech, protože řetězec jako takový se u společností poskytující služby nebude zásadněji měnit, společnost tak většinou pokrývá jeho velkou část, toto platí minimálně o analyzované společnosti. Jedinou možností jak expandovat tedy není pokrytí větší části hodnotového řetězce, ale vytvoření nového, v rámci kterého budou nabízet nové služby, tedy \textit{expansion of services}.

%Tedy rozšiřování by expansion of services. 

\newpage

\section*{B. Životní cyklus produktů (volitelně BCG matice)}
\label{sec:Zivotni cyklus produktu}
\addcontentsline{toc}{subsection}{\nameref{sec:Zivotni cyklus produktu}}

Autoři bohužel nemají k dispozici přehled o interních tržbách za jednotlivé oblasti, pouze aggregovanou hodnout publikovanou v rozvaze. BCG matice je tak pouze odhadnovaná.\\

Produkty které spolčnost poskytuje buď přímo, nebo které jsou poskytovány v rámci holdingu jsou:\\

\begin{enumerate}
	\item daňové poradenství,
	\item účetnictví,
	\item audit.
\end{enumerate}

\begin{table}[!hbtp]
\centering
\begin{tabular}{cc|cc|}
\cline{3-4}
\multicolumn{1}{l}{} & \multicolumn{1}{l|}{} & \multicolumn{2}{c|}{Relativní tržní podíl v procentech} \\ \cline{3-4} 
\multicolumn{1}{l}{} & \multicolumn{1}{l|}{} & \multicolumn{1}{c|}{Vysoký} & Nízký \\ \hline
\multicolumn{1}{|c|}{} &  & \multicolumn{1}{c|}{\cellcolor[HTML]{67FD9A}} & - daňové poradenství \\
\multicolumn{1}{|c|}{} & Vysoký & \multicolumn{1}{c|}{\cellcolor[HTML]{67FD9A}} & - účetnictví \\
\multicolumn{1}{|c|}{Růst trhu v} &  & \multicolumn{1}{c|}{\cellcolor[HTML]{67FD9A}} & - audit \\ \cline{2-4} 
\multicolumn{1}{|c|}{procentech} &  & \multicolumn{1}{c|}{} & \cellcolor[HTML]{FD6864}{\color[HTML]{000000} } \\
\multicolumn{1}{|c|}{} & Nízký & \multicolumn{1}{c|}{} & \cellcolor[HTML]{FD6864} \\
\multicolumn{1}{|c|}{} &  & \multicolumn{1}{c|}{} & \cellcolor[HTML]{FD6864} \\ \hline
\end{tabular}
\caption[BCG matice]{BCG matice, vlastní zpracování.}
\label{tab:BCG matice}
\end{table}

Autoři odhadují, že veškeré společností nabízené produkty se nacházejí ve svém životním cyklu v části s vysokým růstem, ale nízkým relativním podílem v procentech na trhu, to zejména díky saturovanému trhu a vysoké konkurenci.\\

\section*{C. Životní cyklus firmy}
\label{sec:Zivotni cyklus firmy}
\addcontentsline{toc}{subsection}{\nameref{sec:Zivotni cyklus firmy}}

Společnost měla kolísající obrat i příjmy mezi roky 2010 až 2014, od roku 2014 však udržuje trvalý růst, dle souhrnných tabulek níže, lze dovodit, že se podnik nachází v růstové fázi vývoje, respektive osciluje mezi rapidním růstem a dospělostí v rámci růstové fáze.\\

Od roku 2015 lze pozorovat vsoký růst společnosti, který se mírně zpomalil na přelomu let 2018 a 2019. Za použití SMW lze předpokládat, že společnost pokračuje v růstu i v roce 2020,  ačkoli to kvůli absenci účetní závěrky a výkazu zisku a ztrát není možné jasně potvrdit.\\

\begin{figure}[!hbtp]
	\centering
	\includegraphics[width=0.7\linewidth]{Parts/InternalResources/CHART_Zisk_in_time.pdf}
	\caption[Přehled čistého zisku v čase]{Přehled čistého zisku v čase\footnote{Vlastní zpracování vycházející z výkazů zisku a ztrát podniku za období 2010 až 2019.}}
	\label{fig:Prehled cisteho zisku v case}
\end{figure}

\newpage

Co se týče dalších ukazatelů, tedy růstu obratu a růstu bilanční sumy, z tabulek níže je možné dovodit, že podnik procházel turbulentnějším obdobím mezi roky 2010 až 2014, v rámci kterého docházelo ke snižování a následnému zvyšování všech sledovaných metrik. Tyto se stabilizovaly od roku 2015, nicméně je možné sledovat stále zásadní výkyvy v rámci bilanční sumy, která oproti ostatním metrikám v roce 2017 poklesla, pravděpodobným důvodem je pokles růstu čistého zisku v této době a tedy nutnost krýt určité náklady právě hodnotou z bilanční sumy.\\

V kontextu růstu společnosti ve vztahu k růstu HDP není možné vysledovat zásadní korelaci, mimo roky 2010 až 2014 je však viditelné, že společnost rystla procentuálně rychleji než česká ekonomika.\\

\begin{table}[h!]
\centering
\begin{tabular}{l|c|c|c|c|c|}
\cline{2-6}
 & 2010 & 2011 & 2012 & 2013 & 2014 \\ \hline
\multicolumn{1}{|l|}{Obrat} & 67 974 & 64 259 & 61 174 & 63 775 & 63 594 \\ \hline
\multicolumn{1}{|l|}{Růst obratu} & 6 814 & -3 715 & -3 085 & 2 601 & -181 \\ \hline
\rowcolor[HTML]{C0C0C0} 
\multicolumn{1}{|l|}{\cellcolor[HTML]{C0C0C0}Procentuální růst obratu} & 11,1413\% & -5,4653\% & -4,8009\% & 4,2518\% & -0,2838\% \\ \hline
\multicolumn{1}{|l|}{Čistý zisk} & 11 690 & 13 399 & 9092 & 12 622 & 10 314 \\ \hline
\multicolumn{1}{|l|}{Růst čistého zisku} & 4 728 & 1 709 & -4 307 & 3 530 & -2 308 \\ \hline
\rowcolor[HTML]{C0C0C0} 
\multicolumn{1}{|l|}{\cellcolor[HTML]{C0C0C0}Procentuální růst čistého   zisku} & 67,9115\% & 14,6193\% & -32,1442\% & 38,8253\% & -18,2855\% \\ \hline
\multicolumn{1}{|l|}{Bilanční suma} & 38581 & 30801 & 27022 & 30758 & 29937 \\ \hline
\multicolumn{1}{|l|}{Růst bilanční sumy} & 6506 & -7780 & -3779 & 3736 & -821 \\ \hline
\rowcolor[HTML]{C0C0C0} 
\multicolumn{1}{|l|}{\cellcolor[HTML]{C0C0C0}Procentuální růst bilanční   sumy} & 20,2837\% & -20,1654\% & -12,2691\% & 13,8258\% & -2,6692\% \\ \hline
\multicolumn{1}{|l|}{Růst HDP} & 0,8156\% & 1,7461\% & 0,7325\% & 1,3207\% & 4,8969\% \\ \hline
\end{tabular}
\caption[Vývoj sledovaných ukazatelů za roky 2010 - 2014]{Vývoj sledovaných ukazatelů za roky 2010 - 2014\footfullcite[Účetní závěrky za roky 2010 - 2014]{proxy_justice}\textsuperscript{,}\footfullcite{noauthor_hruby_nodate}.}
\label{tab:Vyvoj sledovanych ukazatelu one}
\end{table}

\begin{table}[!hbtp]
\centering
\begin{tabular}{l|l|l|l|l|l|}
\cline{2-6}
 & 2015 & 2016 & 2017 & 2018 & 2019 \\ \hline
\multicolumn{1}{|l|}{Obrat} & 67 197 & 72 345 & 76 690 & 92 918 & 104 449 \\ \hline
\multicolumn{1}{|l|}{Růst obratu} & 3 603 & 5 148 & 4 345 & 16 228 & 11 531 \\ \hline
\rowcolor[HTML]{C0C0C0} 
\multicolumn{1}{|l|}{\cellcolor[HTML]{C0C0C0}Procentuální růst obratu} & 5,6656\% & 7,6611\% & 6,0059\% & 21,1605\% & 12,4099\% \\ \hline
\multicolumn{1}{|l|}{Čistý zisk} & 11 054 & 12 619 & 13 181 & 18 063 & 22 571 \\ \hline
\multicolumn{1}{|l|}{Růst čistého zisku} & 740 & 1 565 & 562 & 4 882 & 4 508 \\ \hline
\rowcolor[HTML]{C0C0C0} 
\multicolumn{1}{|l|}{\cellcolor[HTML]{C0C0C0}Procentuální růst čistého   zisku} & 7,1747\% & 14,1578\% & 4,4536\% & 37,0382\% & 24,9571\% \\ \hline
\multicolumn{1}{|l|}{Bilanční suma} & 31164 & 36115 & 33632 & 36739 & 44179 \\ \hline
\multicolumn{1}{|l|}{Růst bilanční sumy} & 1227 & 4951 & -2483 & 3107 & 7440 \\ \hline
\rowcolor[HTML]{C0C0C0} 
\multicolumn{1}{|l|}{\cellcolor[HTML]{C0C0C0}Procentuální růst bilanční   sumy} & 4,0986\% & 15,8869\% & -6,8753\% & 9,2382\% & 20,2510\% \\ \hline
\multicolumn{1}{|l|}{Růst HDP} & 6,5173\% & 3,6092\% & 6,7425\% & 5,8273\% & 6,9605\% \\ \hline
\end{tabular}
\caption[Vývoj sledovaných ukazatelů za roky 2015 - 2019]{Vývoj sledovaných ukazatelů za roky 2015 - 2019\footnote{Tamtéž.}.}
\label{tab:Vyvoj sledovanych ukazatelu two}
\end{table}

\newpage

\section*{D. Organizační struktura a kultura podniku}
\label{sec:Organizacni struktura}
\addcontentsline{toc}{subsection}{\nameref{sec:Organizacni struktura}}

Organizační strukturu bohužel společnost nezveřejnila, autoři proto při následujících předkládaných informací budou vycházet především ze své zkušenosti pramenicí z bývalé spolupráce se společností.\\

Základní organizační strukturu tvoří představenstvo, manažer oblasti, seniorní zaměstnanec a zaměstnanec.\\

Organizační struktura podniku by se dala zobrazit následovně:\\

\inputfile{Parts/InternalResources/DIAGRAM_Organizational_structure}

Společnost je organizována hierarchicky, nestaví tedy na novějším hedónistickém principu, přestou nejsou zaměstnanci v hierarchii podniku daleko od vedení a přes svého tým leadera (v diagramu zobrazen jako senior), se tak mohou jednodušše obracet na vedení podniku. Mělčí organizační struktura zajišťuje, že se budou především lépe přenášet informace a společnost bude více organizovanější, neboť sami zaměstnanci jsou blízko vedení podniku.\\

Nelze však říci, že výše uvedená organizační struktura je ideální, proto autoři v části 4 rovněž načrtávají doporučení pro jeho změnu, cílící na zefektivnění práce ve společnosti.\\

Co se týče využívání PIS v rámci organizace, společnost používá software třetích stran, pro udržení knowledebasu jsou používány nástroje od Microsoftu, jako je Sharepoint, dále je používán Microsoft Office jako office suite a program Helios pro ostatní PIS, ten je také zároveň používán pro správu práce s daněmi a účetnictví v rámci činnosti podniku\footnote{Společnost používá i mnoho dalších systému, vyřčený přehled je tak spíše demonstrativní.}.\\

Kompetence jsou rozdělovány podle vrio matice, kterou společnost používá, ta zahrnuje jednotlivé bolastni, za které jsou zaměstnanci zodpovědní, zaměstnanci mají rovněž k dispozici obsáhlou metodiku a popsané procesy, na kompetenčním modelu se zásadně podepisují zákonné nároky kladené na změstnance vykonávající účetní, nebo daňovou činnost, tyto jsou do kompetenčního modelu promítnuté (viz problémy spol, legislativní nároky viz výše, kapitola 1.). Dále má směrnice definijucí pracovní náplně. Mají benefit model a výkonostní sledování HR scoreboard, ale společnost svým zaměstnanců vkládá i velkou důveru, kvůli nutnosti auditu se veškerá práce loguje a monitoruje.\\

Autoři se domnívají, že podnik nemá přespříliš manažerů, ale že zvolený přístup, že je jeden manažer zodpovědný za každou oblast je OK, efektivní, není moc micromanagement, na úrovni jednotlivých týmů se uplatňuje i více holistický přístup a seniorní kolega tým leader je furt jenom kolega, za kterým si ostatní kolegové mohou přijít pro radu, autoři považují pracovní procesy a organizaci podniku za efektivní, i když se domnívají, že dodatečným zploštěním organizační struktury by se mohla efektivita potenciálně zvýšit.\\

%Zaměstnanci při svojí práci používají nástroje především 

%Oranizovanost ok, dobře se předávájí informace, zaměstnanci jsou blízko vedení.

%Seniorové, manažeři se mohou prolínat.

\newpage

\section*{E. Kapitálová struktura podniku}
\label{sec:Kapitalova struktura podniku}
\addcontentsline{toc}{subsection}{\nameref{sec:Kapitalova struktura podniku}}

\begin{figure}[!htbp]
  \centering
  \begin{minipage}[b]{0.49\textwidth}
    \includegraphics[width=1\linewidth]{Parts/InternalResources/CHART_Structure_of_assets}
    \caption[Struktura majektu]{Struktura majektu\footfullcite{proxy_merk}}
    \label{fig:Struktura majetku}
  \end{minipage}
  \hfill
  \begin{minipage}[b]{0.49\textwidth}
    \includegraphics[width=1\linewidth]{Parts/InternalResources/CHART_Structure_of_liabilities}
    \caption[Struktura závazků]{Struktura závazků\footnote{Tamtéž.}}
    \label{fig:Struktura zavazku}
  \end{minipage}
\end{figure}

\newpage

\begin{figure}[!hbtp]
	\centering
	\includegraphics[width=0.5\linewidth]{Parts/InternalResources/CHART_Current_assets}
	\caption[Struktura aktiv]{Struktura aktiv\footnote{Tamtéž.}}
	\label{fig:Struktura aktiv}
\end{figure}

Trendově není u dlouhodobých aktiv vidět zásadnější změny, pohyblivější je dlouhodobý hmotný majetek.\\

Co se týče růstu vlastního kapitálu, ten je shrnut v následující tabulce:

\begin{table}[!hbtp]
\centering
\begin{tabular}{|l|c|}
\hline
Datum & Vlastní kapitál \\
\hline
2014 & 22 844 \\
\hline
2015 & 23 898 \\
\hline
2016 & 25 517 \\
\hline
2017 & 17 236 \\
\hline
2018 & 22 300 \\
\hline
2019 & 26 871 \\
\hline
\end{tabular}
\caption[Růst vlastního kapitálu]{Růst vlastního kapitálu\footfullcite[Účetní závěrky za roky 2014 - 2019.]{proxy_justice}}
\label{tab:Rust vlastniho kapitalu}
\end{table}

Vlastní kapitál má rostoucí tendenci a mimo rok 2017, kdy poklesl, jeho výše konstantně rostla.\\

\newpage

\section*{F. Akademické analýzy}
\label{sec:Akademicke analyzy}
\addcontentsline{toc}{subsection}{\nameref{sec:Akademicke analyzy}}

\textit{Our study has produced the following stylized findings. First, industry leaders or “centers of gravity” in terms of scale and scope rose to dominance first and foremost through their accumulation of superior talent, and acquisition of underperforming tangible assets. Second, firms who became centers of gravity had at their core, top management teams who exhibited stable shared leadership. Harking back to Penrose, firms with single leadership (either throughout their business history or during episodic periods) often had able leaders. All but one (Godo) were nonetheless constrained because they “simply don’t have the necessary number of men of the required caliber around” (Penrose, 1959, p. 63). Complementarities in managerial talent across expertise domains through stable shared leadership at the helm enabled firms to grow by pursuing talent and scale acquisition, product expansion and downstream integration. Third, smooth transition to long-term stability of shared leadership was the exception, not the rule. Only one firm (Mie) among over a hundred that entered cotton spinning during out time frame was able to do so. Rather, for the overwhelming majority of firms, TMTs under shared leadership evolved through distinct alternative paths resulting from discord-induced departures. Thus, maintaining stable shared leadership was a difficult endeavor.}\footfullcite{agarwal_centers_2019}

\newpage

\section*{G. V-R-I-O}
\label{sec:V-R-I-O}
\addcontentsline{toc}{subsection}{\nameref{sec:V-R-I-O}}

\inputfile{Parts/IntroductionResources/TABLE_Konkurencni_vyhoda}

- 1. Značka
- 6. Finanční zdroje
- 5. Oblasti zaměření
- 3. Know-how
- 2. Kvalita
- 4. Specializace
- 9. Administrativa
- 7. Marketing
- 8. Lidské zdroje

\noindent\textbf{Značka}\\

Značka je pro firmu trvale udržitelnou výhodou, neboť ze zákona jí může mít pouze ona.\\

\noindent\textbf{Kvalita}\\

Společnost si zakládá na vysoké kvalitě poskytovaných služeb.\\

\noindent\textbf{Know-how}\\

Začátek podnikání v oblasti účetnictví a daní je těžký vzhledem k legislativním nárokům na vykonávanou činnost a know-how, firma je na trhu dlouho, pro má v těchto oblastech konkurenční výhodu.\\

\noindent\textbf{Specializace, oblasti zaměření}\\

Společnost mí širokou specializaci finančních služeb.\\

\noindent\textbf{Finanční zdroje}\\

Společnost má k dispozici vysoké finanční zdroje, nejsou však zcela využity.\\

\noindent\textbf{Marketing}\\

Společnost se na marketing zásadněji nezaměřuje, spoléhá se na organickou akvizici.\\

\noindent\textbf{Lidské zdroje}\\

Společnost má dobře pokryté lidské zdroje, vzhledem ke stoupajícímu počtu živnostníků v oboru účetnictví a daní na trhu, to však nemusí být konkurenční výhoda.\\

\noindent\textbf{Administrativa}\\

Běžná administrativa je prováděna standardnímy způsoby, byla by zde potřeba modernizovat procesy.\\
