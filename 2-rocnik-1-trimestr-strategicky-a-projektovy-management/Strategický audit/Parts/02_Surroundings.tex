\newcommand\SkenovaniVnejsihoOkoli{\texorpdfstring{\MakeUppercase{\romannumeral 2}.}{} SKENOVÁNÍ VNĚJŠÍHO OKOLÍ}
\begin{lefttextpipe}
	{\Large \SkenovaniVnejsihoOkoli}
\end{lefttextpipe}
\phantomsection
\addcontentsline{toc}{section}{\SkenovaniVnejsihoOkoli}

V rámci vnější analýzy se autoři zaměřují na 5 oblastí ze STEEP analýzy, tedy na oblastni sociologické, technologické, ekonomické, ekologické a politicky-právní.\\

%\vspace*{1mm}

%Těmito oblastmi jsou:

%\begin{enumerate}
%	\itemsep0em
%	\item sociologická,
%	\item technologická,
%	\item ekonomická,
%	\item ekologická,
%	\item politicky-právní.	
%\end{enumerate}

%\vspace*{1mm}

\textbf{Sociologická analýza}\\

Co se týče analýzy sociologických dopadů na společnost, je dle autorů třeba vyhodnotit především obecné socio-kulturní faktory v čele s demografickou analýzou. Autoři zvolili pro tuto část následující faktory: počet obyvatelstva, jeho složení podle věku, pohlaví, vzdělání a sociálního postavení, dále kulturní, náboženské a obecné změny v homogenosti/rozmanitosti obyvatelstva.\\

Vzhledem k historickému trendu vývoji obyvatelstva (průběžný růst) od roku 1993 nelze předpok\-ládat, že by se v nejbližších letech trend výrazněji změnit. Z tohoto důvodu autoři nepovažují tento sociologický faktor za potenciálně rizikový, naopak z růstu počtu obyvatel (Viz Figure 3) a rostoucího počtu živností\footnote{Autoři zde pracují s předpokladem, že subjekty provozující živnostenské podnikání představují velkou část potenciální klientely společnosti.}\textsuperscript{,}\footnote{Vzhledem k absenci dat pro rok 2021 a probíhajíci pandemii však není vyloučeno, že pro rok 2021 má graf již sestupnou tendenci, v takovémto případě by bylo třeba daný trend vyhodnotit jako rizikový pro společnost Proxy a.s.} (viz Figure 4) se autoři domnívají, že pro společnost by tento demografický trend (zvyšování počtu obyvatel, zvyšování počtu podnikajících subjektů) mohl představ\-ovat příznivý faktor pro její vlastní ekonomický růst.\\ 

\begin{figure}[!hbtp]
	\centering
	\includegraphics[width=0.8\linewidth]{Parts/InternalResources/CHART_Population_in_time.pdf}
	\caption[Vývoj počtu obyvatel v čase]{Vývoj počtu obyvatel v čase\footfullcite{noauthor_obyvatelstvo_nodate}}
	\label{fig:Vyvoj poctu obyvatel v case}
\end{figure}

\newpage

\begin{figure}[!hbtp]
	\centering
	\includegraphics[width=0.8\linewidth]{Parts/InternalResources/CHART_Zivnosti_in_time.pdf}
	\caption[Vývoj počtu živností v čase]{Vývoj počtu živností v čase\footfullcite{noauthor_rocni_nodate}}
	\label{fig:Vyvoj poctu zivnosti v case}
\end{figure}

\newpage

\section*{A. Porterova analýza}
\label{sec:Porterova analyza}
\addcontentsline{toc}{subsection}{\nameref{sec:Porterova analyza}}