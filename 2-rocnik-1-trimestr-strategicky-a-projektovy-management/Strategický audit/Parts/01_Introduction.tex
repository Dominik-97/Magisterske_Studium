\maketitle
\vspace*{5mm}
\tableofcontents

\newpage

\thispagestyle{fancy}

\newcommand\PredstaveniSpolecnosti{PŘEDSTAVENÍ SPOLEČNOSTI Proxy, a.s.}
\begin{lefttextpipe}
	{\huge \PredstaveniSpolecnosti}
\end{lefttextpipe}
\phantomsection
\addcontentsline{toc}{section}{\PredstaveniSpolecnosti}

Předmětem podnikání společnosti je především \textit{daňové poradenství} a \textit{činnost účetních poradců, vedení účetnictví, vedení daňové evidence}. Společnost má okolo 60 zaměstnanců\footnote{V Merku je možné dohledat číslo 50 - 99 zaměstnanců.}\textsuperscript{, }\footfullcite{proxy_merk} a je lokalizována v České republice. Společnost je součástí většího holdingu PROXY Holding a.s.. Společnost je dělená na několik oddělení, z nichž každá se soustředí na specifickou oblast v rámci daňového a účetního poradentsví, každá z těchto oblastní má vlastního manažera. Obecné směřování společnosti je stanovováno představenstvem.\\

Společnost se pohybuje výhradně na českém trhu v oblastni daňové a účetní, jejímy klienty jsou subjekty, které v České republice vykonávají jakoukoli ekonomickou činnost v rámci které jsou povinny řídit se českými daňovými a účetnímy předpisy. Hlavní službou společnost Proxy, a.s. je tak poskytování služeb související právě s činnostmi týkající se daní a vedení účetnictví a jedná se tedy výhradně o účetní a daňovou společnost.

\section*{A. Současná vize a poslání podniku}
\label{sec:Vize a poslani}
\addcontentsline{toc}{subsection}{\nameref{sec:Vize a poslani}}

\subsection*{A.1 Poslání a mise společnosti}
\label{sec:Poslani}
\addcontentsline{toc}{subsubsection}{\nameref{sec:Poslani}}

Dle informací na webových stránkách společnosti je základní poslání a mise poskytování služeb na poli daňového poradenství, účetnictví, mzdové evidence a dalšího specializovaného poradenství\footfullcite{proxy_website}, především pro zahraniční investory vstupující na český trh a zjednodušit a zpřehlednit jim tam investici a fungování v České republice. 

\subsection*{A.2 Vize}
\label{sec:Vize}
\addcontentsline{toc}{subsubsection}{\nameref{sec:Vize}}

Vize společnosti spočívá především v tom, aby se stala partnerem pro své klienty a nadále je podporovala kvalifikovaným poradenstvím a to jak pro českou, tak i pro mezinárodní klientelu. Společnost se stala v roce 2004 součástí mezinárodní asociace poradenských firem HLB se sídlem v Londýně a nadále tak podporuje svojí vizi o kvalitní poradenské činnosti v dané oblasti\footfullcite{proxy_o_firme}.

\subsection*{A.3 Aktuální problémy}
\label{sec:Problemy}
\addcontentsline{toc}{subsubsection}{\nameref{sec:Problemy}}

%Problémy jsou příležitosti.

%Problém bude škláování a legislativní požadavky a jejich implementace a přenesení do procesů týkající se poradenství v oblasti daní a činnosti daňových poradců a účetních poradců a udržování knowledgebasu.

%Dále školení lidí, tato oblast je dost problematická, rovněž taky shánění lidí?

S ohledem na informace na webové stránce společnosti a finanční výkazy (viz analýza níže) se společnost nepotýká s nedostatkem příležitostí a z nichž pramenícího zisku. Nejzásadnějším problémem je škálování vzhledem k dlouhodobému ekonomickému růstu a větší poptávce po daňovém a účetním poradenství, toto lze dovodit s ohledem jejich stránku kde chtějí studenty a mají permanentně vyvěšené pracovní nabídky, ČSÚ a vývoj počtu podnikatelů v dané oblastni a růstu vykazovanému v účetniství. Lze soudit ze statistik MPO odkaz za sledované období čtvrtého, respektive třetího kvartálu od roku 2016 až 2021

Ahoj.

\inputfile{Parts/IntroductionResources/Pocet_Zivnost_V_Prubehu_Casu}
Data2 2

Hello?
\inputfile{Parts/IntroductionResources/Vyvoj_zisku}

%\input{Parts/IntroductionResources/Vyvoj_zisku}

Taky bych sem měl zařadit analýzu, respektive graf rostoucích příjmů - pak rovněž odkázat na část, kde řeším příjmy v analýze.

Lze možná dooknce usoudit, že nárůst je způsobený i růstem příležitostí během COVIDU 19 a sním spojené nutnosti postarat se o daně a účetnivtví.


 a udržování knowledge basu a přizpůsobování procesů neustále se měnícím požadavkům na daňové a účetní, respektive finanční poradenství a povinnosti s ním související v českém právním prostoru, jedná se tedy o zásadní externí faktor.

\subsection*{A.4 Green deal}
\label{sec:Green deal}
\addcontentsline{toc}{subsubsection}{\nameref{sec:Green deal}}

Green deal je, odkaz, vztah společnosti ke green dealu je, blabla odkaz. Moc nemusí řešit green deal, nejsou výrobní společnost, která by generovala velké množství znečištění.

\section*{B. Identifikace procesu strategického řízení}
\label{sec:Identifikace procesu strategickeho rizeni}
\addcontentsline{toc}{subsection}{\nameref{sec:Identifikace procesu strategickeho rizeni}}

Úplně nemají, jenom základy podle kterých se řídí a jsou aktualizovány vedením společnosti, jsou v relativně stabilním businessu a pokrývají všechny oblasti v dané kategorii v četně auditu, takže úplně nepotřebují strategický plán.

Majá spíš operativní - na koho, kdy a jak se zaměří - maj i svojí klientelu, takže nepotřebují akvizici tak zásadně.

\section*{C. Popis současného obchodního modelu}
\label{sec:Popis soucasneho obchodniho modelu}
\addcontentsline{toc}{subsection}{\nameref{sec:Popis soucasneho obchodniho modelu}}

Současný model se soustředí na poskytování služeb v oblasti daňové a účetní a auditorské (v rámci jiné společnosti).

\newpage

\inputfile{Parts/IntroductionResources/Table_1}

\section*{D. Současné cíle společnosti}
\label{sec:Soucasne cile spolecnosti}
\addcontentsline{toc}{subsection}{\nameref{sec:Soucasne cile spolecnosti}}

\subsection*{D.1 Strategické}
\label{sec:Strategicke}
\addcontentsline{toc}{subsubsection}{\nameref{sec:Strategicke}}

Asi ovládnout trh.

\subsection*{D.2 Taktické}
\label{sec:Takticke}
\addcontentsline{toc}{subsubsection}{\nameref{sec:Takticke}}

Asi takticky ovládnout trh.

\section*{E. Odhad diskontních faktorů}
\label{sec:Odhad diskontnich faktoru}
\addcontentsline{toc}{subsection}{\nameref{sec:Odhad diskontnich faktoru}}

Haha test.

Test \footfullcite{noauthor_fideicommissum_nodate}

$EVA_t = NOPAT_t - WACC_t$ (aktiva celkem$_t -$ krátkodobé závazky$_t$)

\inputfile{Parts/IntroductionResources/Table_2}