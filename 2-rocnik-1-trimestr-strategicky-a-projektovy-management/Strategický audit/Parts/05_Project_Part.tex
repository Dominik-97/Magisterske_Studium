\begin{lefttextpipe}
	{\Large \MakeUppercase{\romannumeral 5} PROJEKTOVÁ ČÁST}
\end{lefttextpipe}