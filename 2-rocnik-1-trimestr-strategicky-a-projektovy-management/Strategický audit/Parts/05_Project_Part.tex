\newcommand\ProjektovaCast{\texorpdfstring{\MakeUppercase{\romannumeral 5}.}{} PROJEKTOVÁ ČÁST}
\begin{lefttextpipe}
	{\Large \ProjektovaCast}
\end{lefttextpipe}
\phantomsection
\addcontentsline{toc}{section}{\ProjektovaCast}

Vzhledem k informacím uvedeným v analýze, ze kterých vyplývá, že společnost Proxy a.s. dlouhodobě prosperuje a udržuje si ekonomický růst, přičmež se nepotýká se zásadnějšími problémy (viz kapitola A.3 Aktuální problémy) se autoři rozhodli, že navrhnutý projekt se bude primárně týkat implementace procesů ve společnosti, které povedou k zavedení společenské odpovědnosti společnosti vzhledem k životnímu prostředí (viz kapitola A.4 Green deal), neboť společnost Proxy a.s. se, jak již bylo uvedeno, touto aktivitou aktivně nezabývá.\\

Zavedení procesů směřujících k alespoň zmírnění negativních dopadů na životní prostředí, nebo dokonce k příspění ke zlepšení životního prostředí pro všechny je jakousi morální obligací vedení každého podnikajícího subjektu, Proxy a.s. by neměl být výjimkou, to že se Proxy a.s. aktivně nezajímá o ochranu přírodního bohatství je dle pohledu autorů práce alarmují a mělo by se to v co nejkratší době změnit, a to i přes to, že společnost Proxy a.s. zabývající se výhradně poskytováním služeb není zdaleka největším přispěvatelem ke znečišťování na českém trhu.\\

Domníváme se dokonce, že aktivní participace na ochraně životního prostředí by přispěla nejen k blahobytu všech obyvatel, ale i k dobrému jménu společnosti, což se v konečném důsledku může odrazit na její konkurenční výhodě a může přilákat nové potenciální klienty, pro které je ochrana přírody velice zásadní otázkou, neboť jak již bylo vyřčeno, ochrana přírody je v poslední době téme, které přibírá na důležitosti a intenzitě velkou měrou.\\

Autoři jsou si vědomi, že vzhledem k momentální situaci se zřejmě nedá rojekt tohoto charakteru zcela charakterizovat jako strategicky důležité opatření, ze které bude mít firma okamžitý prospěch, to však neznamená, že není společensky potřebné, a to i když může být v prvopočátku implementace tohoto projektu hlavním příjemncem jeho benefitů obyvatelstvo jako nepřímý důsledek zlepšení životního prostředí, místo "\textit{kasy}" podniku, které to ale, jak již bylo řečeno výše, rovněž může pozitivně ovlivnit.\\

\textbf{V jaké kvalitě (základní parametry) dané opatření musí být zavedeno, aby byl příjemnce benefitu z něho plynoucího spokojen} Kvalitativně dané opatření musí splňovat několik hlavních kritérií, musí být SMART, mít reálný dopad a u společnosti musí být vidět, že ho dělá, aby si například pomohla i marketingově.\\

Tato změna se tak musí odrazit v několika oblastech:

\begin{itemize}
	\item procurementu\footfullcite{anon_sustainable_2013}\textsuperscript{,}\footfullcite{anon_using_2011} - především v rámci nákupu a používání věcí, které jsou vyrobeny z obnovitelných zdrojů a svojí výrobou neškodí přírodě,
	\item dopravě\footfullcite{noauthor_co2_2019} - není možné rozšiřovat flotilu automobilů, společnost by měla podporovat sdílenou mobilitu a podporovat MHD,
	\item a facility managementu\footfullcite{jayasena_environmental_2019} - především ve smyslu spotřeby energií z obnovitelných zdrojů a přizpůsobení nemovitých věcí nejnovějším a nejšětrnějším standardům a likvidace odpadů,
	\item zapojení se do vnějších enviromentálních projektů,
	\item marketing.
\end{itemize}

Co se rozpočtu na změnu týče, předpoklad je takový, že náklady na implementaci nebudou příliš vysoké v počáteční fázi. V rámci procurementu se jedná především o nakupování recyklovaných, netoxických, kompostovatelných, či biodegradabel, waste free předmětů. Dále například \textit{low power} elektrického příslušenství, většího využití solární energie, udržitelná obměna mobiliáře a podobně. Bez pochyby by se náklady na procurement zvýšily, nicméně s ohledem na to, že \textit{eco-friendly} produkty jsou čím dál tím dostupnější, přičemž jejich cena dlouhodobě kleasá.\\

V rámci dopravy je naopak možné uvažovat i o úspoře, místo nutnosti hradit povinné ručení, havarijní pojištění, pohonné hmoty a další náklady spojené s údržbou vozového parku, mohla by se společnost zaměřit například na redukovaný, avšak sdílený vozový park elektro automobilů, příspěvky na hromadnou dopravu, větší podporu homeoffice (s tím spojené omezení využívání kancelářských prostor) a podporu alternativních způsobů dopravy.\\

V rámci linkvidace je zásadní se zaměřit na třídění odpadů a správnou likvidaci nebezpečných produktů - například ekologická likvidace počítačů. Ačkoli je s ekologickou likvidací spojena vyšší cena, tato není vzhledem k ziskům společnosti absolutně relevantní.\\

V rámci zapojení se do vnějších enviromentálních procesů není možné náklady předem vyčíslit, odhat se příliš odvíjí od druhů projektů. Marketing rovněž není jednoduché odhadnout, v případě nejběžnějších kanálů lze však začít s již několika tisíci.\\

Implementace těchto změn povede nejen k přispění společnosti k udržitelnosti, ale rovněž přispěje k jejímu dobrému ménu a může tak generovat potenciální nové příležitosti.\\

\newpage

\section*{A. WBS tabulka a diagram projektu}
\label{sec:WBS}
\addcontentsline{toc}{subsection}{\nameref{sec:WBS}}

\begin{center}
\begin{figure}[!hbtp]
\resizebox{14cm}{10cm}{%
\begin{tikzpicture}[
  level 1/.style={sibling distance=40mm},
  edge from parent/.style={->,draw},
  >=latex]

% root of the the initial tree, level 1
\node[root] {Green Deal projekt}
% The first level, as children of the initial tree
  child {node[level 2] (c1) {Procurement}}
  child {node[level 2] (c2) {Doprava}}
  child {node[level 2] (c3) {Facility management}}
  child {node[level 2] (c4) {Vnější projekty}}
  child {node[level 2] (c5) {Marketing}};

% The second level, relatively positioned nodes
\begin{scope}[every node/.style={level 3}]
\node [below of = c1, xshift=18pt, yshift=-5pt] (c11) {Výběr dodavatele};
\node [below of = c11, yshift=-4pt] (c12) {Změna dodavatele};
\node [below of = c12, yshift=-10pt] (c13) {Nákup eco-friendly produktů};

\node [below of = c2, xshift=15pt, yshift=-12pt] (c21) {Rozprodej stávajícího vozového parku};
\node [below of = c21, yshift=-30pt] (c22) {Zavedení příspěvku na hromadnou, sdílenou dopravu};
\node [below of = c22, yshift=-36pt] (c23) {Implementace alternativních způsobů dopravy};
\node [below of = c23, yshift=-18pt] (c24) {Nákup elektro automobilů};
\node [below of = c24, yshift=-18pt] (c25) {Vytvoření malého sdíleného vozového parku};

\node [below of = c3, xshift=15pt, yshift=-7pt] (c31) {Research dotací};
\node [below of = c31, yshift=-11pt] (c32) {Ekologická likvidace odpadů};
\node [below of = c32, yshift=-23pt] (c33) {Přizpůsobení nejvyšším enviromentálním standardům};
\node [below of = c33, yshift=-29pt] (c34) {Spotřeba energií z obnovitelných zdrojů};

\node [below of = c4, xshift=15pt] (c41) {Research};
\node [below of = c41, yshift=6pt] (c42) {Zapojení};

\node [below of = c5, xshift=15pt] (c51) {Definice};
\node [below of = c51, yshift=6pt] (c52) {Research};
\node [below of = c52, yshift=6pt] (c53) {Kampaně};

%\node [below of = c4, xshift=15pt] (c41) {Default arrows};
%\node [below of = c41] (c42) {Arrow library};
%\node [below of = c42] (c43) {Resizing tips};
%\node [below of = c43] (c44) {Shortening};
%\node [below of = c44] (c45) {Bending};
\end{scope}

% lines from each level 1 node to every one of its "children"
\foreach \value in {1,2,3}
  \draw[->] (c1.195) |- (c1\value.west);

\foreach \value in {1,...,5}
  \draw[->] (c2.195) |- (c2\value.west);

\foreach \value in {1,...,4}
  \draw[->] (c3.195) |- (c3\value.west);
  
\foreach \value in {1,2}
  \draw[->] (c4.195) |- (c4\value.west);
  
\foreach \value in {1,...,3}
  \draw[->] (c5.195) |- (c5\value.west);
  
%\foreach \value in {1,...,5}
%	\draw[->] (c4.195) |- (c4\value.west);
\end{tikzpicture}
}
\caption[WBS diagram projektu]{WBS diagram projektu, vlastní tvorba.}
\label{fig:WBS diagram projektu}
\end{figure}
\end{center}

\newpage

\begin{table}[!hbtp]
\centering
\begin{tabular}{|l|l|l|}
\hline
WBS úroveň & ID úkolu & Aktivita \\ \hline
0 (Hlavní) & - & Zavedení Green Dealu \\ \hline
\rowcolor[HTML]{C0C0C0} 
I (Balíček) & 1 & Procurement \\ \hline
II (Aktivita) & \ \ 1.1 & Výběr dodavatele \\ \hline
II (Aktivita & \ \ 1.2 & Změna dodavatele \\ \hline
II (Aktivita) & \ \ 1.3 & Nákup eco-friendly produktů \\ \hline
\rowcolor[HTML]{C0C0C0} 
I (Balíček) & 2 & Doprava \\ \hline
II (Aktivita) & \ \ 2.1 & Rozprodej stávajícího vozového parku \\ \hline
II (Aktivita) & \ \ 2.2 & Zavedení příspěvku na hromadnou, sdílenou dopravu \\ \hline
II (Aktivita) & \ \ 2.3 & Implementace alternativních způsobů dopravy \\ \hline
II (Aktivita) & \ \ 2.4 & Nákup elektro automobilů \\ \hline
II (Aktivita) & \ \ 2.5 & Vytvoření malého sdíleného vozového parku \\ \hline
\rowcolor[HTML]{C0C0C0} 
I (Balíček) & 3 & Facility management \\ \hline
II (Aktivita) & \ \ 3.1 & Research dotací \\ \hline
II (Aktivita) & \ \ 3.2 & Ekologická likvidace odpadů \\ \hline
II (Aktivita) & \ \ 3.3 & Přizpůsobení nejvyšším enviromentálním standardům \\ \hline
II (Aktivita) & \ \ 3.4 & Spotřeba energií z obnovitelných zdrojů \\ \hline
\rowcolor[HTML]{C0C0C0} 
I (Balíček) & 4 & Vnější projekty \\ \hline
II (Aktivita) & \ \ 4.1 & Research \\ \hline
II (Aktivita) & \ \ 4.2 & Zapojení \\ \hline
\rowcolor[HTML]{C0C0C0} 
I (Balíček) & 5 & Marketing \\ \hline
II (Aktivita) & \ \ 5.1 & Definice \\ \hline
II (Aktivita) & \ \ 5.2 & Research \\ \hline
II (Aktivita) & \ \ 5.3 & Kampaně \\ \hline
\end{tabular}
\caption[WBS tabulka projektu]{WBS tabulka projektu, vlastní tvorba.}
\label{tab:WBS tabulka projektu}
\end{table}

% k výše uvedenému autoři nepokládají za nutné rozšiřovat aktivity společ

\newpage

\section*{B. Rizika tohoto projektu a očekávaný přínos}
\label{sec:Rizika tohoto projektu a ocekavany prinos}
\addcontentsline{toc}{subsection}{\nameref{sec:Rizika tohoto projektu a ocekavany prinos}}

\begin{table}[h]
\centering
\begin{tabularx}{\textwidth}{X|X}
\thickhline
Rizika & Opatření \\ \hline
Neexistující, nebo nevyhovující dodavatelé & Potřeba udělat research marketu před implementací řešení. \\
Změna dodavatele nemusí proběhnout dobře, \newline nespolehlivost dodavatele & Potřeba udělat research a sepsat výhodnou smlouvu. \\
Prodloužení délky dodání. & Implementace procesů směřujících k zefektivnění dodávek (objednávka v dostatečném časovém předstihu). \\
Nezájem o prodávané automobily & Research marketu, připravenost na situaci využití aut do doby jejich odepsání. \\
Nezájem o využívání sdílených prostředků dopravy, sdíleného vozového parku & Probrat možnosti a potenciální zájem o sdílené dopravní prostředky se zaměstnanci před implementací plánu. \\
Absence infrastruktury pro ekologickou likvidaci odpadu & Research před implementací plánu, zjištění, zda by infrastruktura šla doplnit. \\
Přizpůsobení prostor nejvyšším enviromentálním standardům a napojení na obnovitelné zdroje způsobí nedostupnost kanceláří & Nastavení homeofficu vě většině případech, provádět změny mimo pracovní dobu. \\
Neefektivita marketingu & Definovat cílovou skupinu a nejlepší způsoby, jak ji oslovit. \\ \thickhline
\end{tabularx}
\caption{Rizika projektu a opatření}
\label{tab:Rizika projektu a opatreni}
\end{table}