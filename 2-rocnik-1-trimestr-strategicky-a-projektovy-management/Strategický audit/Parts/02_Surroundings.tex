\newcommand\SkenovaniVnejsihoOkoli{\texorpdfstring{\MakeUppercase{\romannumeral 2}.}{} SKENOVÁNÍ VNĚJŠÍHO OKOLÍ}
\begin{lefttextpipe}
	{\Large \SkenovaniVnejsihoOkoli}
\end{lefttextpipe}
\phantomsection
\addcontentsline{toc}{section}{\SkenovaniVnejsihoOkoli}

V rámci vnější analýzy se autoři zaměřují na 5 oblastí ze STEEP analýzy, tedy na oblastni sociologické, technologické, ekonomické, ekologické a politicky-právní.\\

%\vspace*{1mm}

%Těmito oblastmi jsou:

%\begin{enumerate}
%	\itemsep0em
%	\item sociologická,
%	\item technologická,
%	\item ekonomická,
%	\item ekologická,
%	\item politicky-právní.	
%\end{enumerate}

%\vspace*{1mm}

\noindent\textbf{Sociologická analýza}\\

Co se týče analýzy sociologických dopadů na společnost, je dle autorů třeba vyhodnotit především obecné socio-kulturní faktory v čele s demografickou analýzou. Autoři zvolili pro tuto část následující faktory: počet živností, počet obyvatelstva, jeho složení podle věku, vzdělání.\\

Vzhledem k historickému trendu vývoji obyvatelstva (průběžný růst) od roku 1993 nelze předpok\-ládat, že by se v nejbližších letech trend výrazněji změnit. Z tohoto důvodu autoři nepovažují tento sociologický faktor za potenciálně rizikový, naopak z růstu počtu obyvatel (Viz \textit{Figura 3}) a rostoucího počtu živností\footnote{Autoři zde pracují s předpokladem, že subjekty provozující živnostenské podnikání představují velkou část potenciální klientely společnosti.}\textsuperscript{,}\footnote{Vzhledem k absenci dat pro rok 2021 a probíhajíci pandemii však není vyloučeno, že pro rok 2021 má graf již sestupnou tendenci, v takovémto případě by bylo třeba daný trend vyhodnotit jako rizikový pro společnost Proxy a.s.} (viz \textit{Figura 4}) se autoři domnívají, že pro společnost by tento demografický trend (zvyšování počtu obyvatel, zvyšování počtu podnikajících subjektů) mohl představ\-ovat příznivý faktor pro její vlastní ekonomický růst.\\ 

\begin{figure}[!hbtp]
	\centering
	\includegraphics[width=0.8\linewidth]{Parts/InternalResources/CHART_Population_in_time.pdf}
	\caption[Vývoj počtu obyvatel v čase]{Vývoj počtu obyvatel v čase\footfullcite{noauthor_obyvatelstvo_nodate}}
	\label{fig:Vyvoj poctu obyvatel v case}
\end{figure}

\newpage

\begin{figure}[!hbtp]
	\centering
	\includegraphics[width=0.8\linewidth]{Parts/InternalResources/CHART_Zivnosti_in_time.pdf}
	\caption[Vývoj počtu živností v čase]{Vývoj počtu živností v čase\footfullcite{noauthor_rocni_nodate}}
	\label{fig:Vyvoj poctu zivnosti v case}
\end{figure}

Co se týče věku(viz \textit{Figura 5} a vzdělání, autoři se nedomnívají, že vzdělání představu zásadnější faktor pro změnu velikosti potenciální klientely analyzovaného subjektu, v posledních letech je možné sledovat spíše zvětšující se podíl vysokoškolsky vzdělaného obyvatelstva, což by firmě mohlo dokonce prospět\footfullcite{noauthor_obyvatelstvo_nodate-1}\textsuperscript{,}\footnote{Výsledky za sčítání lidu 2021 nejsou ke dni zpracování práce dostupné, viz \url{https://www.czso.cz/csu/scitani2021/vysledky}.}.\\

Autoři však považují za problém stárnutí obyvatelstva, které by mohlo přispět k menší ekonomické aktivitě a tím negativně ovlivnit činnost analyzovaného subjektu\footfullcite{noauthor_starnuti_nodate}.\\

\newpage

\begin{figure}[!hbtp]
	\centering
	\includegraphics[width=0.8\linewidth]{Parts/ExternalResources/CHART_Pocet_obyvatel_dle_veku.pdf}
	\caption[Počet obyvatel dle věku]{Počet obyvatel dle věku\footfullcite{noauthor_statistiky_nodate}.}
	\label{fig:Pocet obyvatel dle veku}
\end{figure}

%Co se týče


%počet, věk, vzdělání

\noindent\textbf{Technologická analýza}\\

Technologie v rámci analyzované společnosti nehrají zásadní roli, k vykonávání činnosti není třeba mít nejnovější technologické vybavení a dodržovat nejnovější technologické postupy.\\

Problémy v oblasti technologického světa\footnote{Nyní je možné pozorovat na světových trzích například nedostatek výpočetních čipů.} tak na společnost nemusí mít zásadnější dopady, pokud se nebude jednat o zásadní problémy s dodávkou jakéhokoliv technologického vybavení, neboť i analyzovaná společnost potřebuje ke svojí činnosti výpočetní technologie.\\

Obecné problémy, které by analyzovanou společnost mohly ovlivnit, i když ne do značné míry jsou nedostatek technologických zařízení, či špatná kvalita dodávaných technologických zařízení.\\

\newpage

\noindent\textbf{Ekonomická analýza}\\

\vspace*{-2mm}

V rámci ekonomické analýzy se autoři zaměřují především na nezaměstnanost, inflaci, růst ekonomiky, kurzové změny a platební bilanci.\\

Všechny výš zmíněné faktory mohou mít na sppolečnost zásadní dopad, proto je třeba je monitorovat a včas podniknout kroky ke zmírnění potenciálních problémů.\\

Nezaměstnanost se drží od 3. čtvrtletí 2020 na úrovni okolo 3 procent, od prvního čtvrtletí roku 2021 začala klesat\footfullcite{noauthor_zamestnanost_nodate} toto pro společnost může mít dvojí vliv, na jednu stranu je nízká míra nezaměstnanosti žádaná a pro obecné směřování ekonomiky žádoucí a má tedy i nepřímo pozitivní vliv i na analyzovanout společnost\footnote{Společnost se službami zaměřuje především na podnikající subjekty, to však nevylučuje spolupráci s fyzickou osobou nepodnikající.}, na druhou stranu může nižší míra nezaměstnanosti, respektive menší míra uchazečů o zaměstnání s tím spojená zapříčit problémy se sháněním zaměstnanců pro společnost.\\

Inflace pro společnost nepředstavuje ten nejzásadnější faktor, neboť má vzhledem ke své činnosti poměrně nízkou výkonovou spotřebu (viz kapitola \textit{C. Životní cyklus firmy}), to samozřejmě neznamená, že jí společnost nemusí sledovat a nemusí brát potaz při vytváření plánů a rozhodování, inflace má totiž dopad na mnoho věcí, mezi nimi například na mzdové náklady a rovněž na celkovou stabilitu ekonomiky, ta pokud by byla nestabilní, by samozřejmě zásadně ovlivnila fungování a činnost společnosti. Od konce třetího měsíce roku 2021 inflace průběžně roste\footfullcite{noauthor_inflace_nodate}, proto je třeba jí sledovat se zvýšenou opatrností.\\

Dále se polečnost musí zaměřovat na mnoho dalších ukazatelů, jako je růst ekonomiky\footfullcite{zurovec_ekonomika_nodate}, společnost má dále mnoho zahraničních klientů, proto jsou pro ní důležitý rovněž kurzové změny společně s platební bilancí, které musí důkladně sledovat.\\

%Přidat nějaký grafy lol.\\

%Poptávka, počet živnostníků, životní situace, růst ekonomika, zájem zahraničních investorů, to vše je může ohrozit. A také mnoho dalšího!\\

%inflace, růst ekonomiky, zájem zahraničních investorů a kurzové změny.

\noindent\textbf{Ekologická analýza}\\

\vspace*{-2mm}

Jak již bylo řečeno v kapitole \textit{A.4 Green deal}, činnost, kterou společnost vykonává, nemá zásadnější dopad na životní prostředí, to ovšem neznemená, že nemá dopad žádný, nebo naopak přispívá ke zlepšování životního přírodí (viz kapitola \textit{V. Projektová část}).\\

Ekologie jako vnější faktor však může mít na společnost zásadní dopady, především nepřímo prostřednictvím negativních dopadů na její klienty pohybující se v odvětvích, která jsou závislá na zdravém životním prostředí.\\

Dopady však mohou být i přímé a to jak negativní, tak i pozitivní při zlepšení stavu životního prostředí.\\

\noindent\textbf{Politicko právní analýza}\\

Společně s ekonomickými problémy je tento bod pro společnost největším rizikem. Česká republika má stabilní právní prostředí a relativně stabilní politickou situaci, autoři tedy nepovažují problémy spojené s náhlou změnou právního nebo politického prostředí za pravděpodobné.\\

Zásadnějším vnějším faktorem, na který si společnost musí dávat pozor jsou však novelizace právních norem. Například zákon č. 586/1992 Sb. o daních z příjmu\footfullcite{noauthor_parlament_nodate}\textsuperscript{,}\footfullcite{noauthor_zakon_1992} a zákon č. 235/2004 Sb. o dani z přidané hodnoty\footnote{Tamtéž.}\textsuperscript{,}\footfullcite{noauthor_zakon_2004} byli v posledních letech souhrnně 29 novelizovány.\\

%, přidat odkaz na ty zákony a na seznam novelizací na stránkách úřadu vlády.to ovšem nez\\

Společnost s těmito změnami musí udržovat krok, aby mohla i nadále poskytovat kvalitní služby a neublížila si na svém dobrém jméně a pověsti.\\

%rapidní změna, novelizace

%\textbf{

\newpage

\section*{A. Porterova analýza}
\label{sec:Porterova analyza}
\addcontentsline{toc}{subsection}{\nameref{sec:Porterova analyza}}

\begin{table}[!hbtp]
\centering
\resizebox{13cm}{!}{%
\begin{tabularx}{\textwidth}{|l|X|X|X|}
\hline
Síla vlivu & Malý & Střední & Velký \\ \hline
\begin{tabular}[c]{@{}l@{}}Zainteresované\\ osoby\end{tabular} & {\color[HTML]{C0C0C0} \footnotesize{Drobní investoři.}} & {\color[HTML]{C0C0C0} } & {\color[HTML]{C0C0C0} \footnotesize{Kontroly ze strany úředníků jsou časté, mohou vznikat problémy. Zásadní vliv má rovněž plánování v rámci holdingu.}} \\ \hline
Dodavatelé & {\color[HTML]{C0C0C0} \footnotesize{Dodavatelé nemají zásadní vliv na fungování společnosti, společnost ke své činnosti nepotřebuje zásadnější dodávky.}} & {\color[HTML]{C0C0C0} } & {\color[HTML]{C0C0C0} } \\ \hline
Odběratelé (Zákazníci) & {\color[HTML]{C0C0C0} \footnotesize{V rámci objednávky mají malý vliv na samotnou nabídku.}} & {\color[HTML]{C0C0C0} } & {\color[HTML]{C0C0C0} \footnotesize{Dodávají dokumenty ke zpracování, problém v případě chybného zpracování.}} \\ \hline
\begin{tabular}[c]{@{}l@{}}Potenciální\\ konkurence\end{tabular} & {\color[HTML]{C0C0C0} } & {\color[HTML]{C0C0C0} } & {\color[HTML]{C0C0C0} \footnotesize{Ano, v této branži není možné se zásadně odlišovat, v případě poškození dobrého jména společnosti může konkurence přebrat velké množství zákazníků. Zároveň počet subjektů nabízející daňové a účetní poradenství stále stoupá.}} \\ \hline
Substituty & {\color[HTML]{C0C0C0} } & {\color[HTML]{C0C0C0} } & {\color[HTML]{C0C0C0} \footnotesize{Ano, činnost účetní, daňová a auditorská je regulována a nemůže se tak zásadně odlišovat, opět záleží na dobrém jméně společnosti, nabízené služby jiných subjektů mají velký dopad na cenotvorbu.}} \\ \hline
\begin{tabular}[c]{@{}l@{}}Odvětvová\\ konkurence\end{tabular} & {\color[HTML]{C0C0C0} } & {\color[HTML]{C0C0C0} } & {\color[HTML]{C0C0C0} \footnotesize{Vysoká, ať už ze strany právnických osob, tak fyzických osob podnikajících.}} \\ \hline
\end{tabularx}
}
\caption[Porterova analýza]{Porterova analýza, vlastní tvorba.}
\label{tab:Porterova analyza}
\end{table}

