\newcommand\SkenovaniVnejsihoOkoli{\texorpdfstring{\MakeUppercase{\romannumeral 2}.}{} SKENOVÁNÍ VNĚJŠÍHO OKOLÍ}
\begin{lefttextpipe}
	{\Large \SkenovaniVnejsihoOkoli}
\end{lefttextpipe}
\phantomsection
\addcontentsline{toc}{section}{\SkenovaniVnejsihoOkoli}

V rámci vnější analýzy se autoři zaměřují na 5 oblastí ze STEEP analýzy, tedy na oblastni sociologické, technologické, ekonomické, ekologické a politicky-právní.\\

%\vspace*{1mm}

%Těmito oblastmi jsou:

%\begin{enumerate}
%	\itemsep0em
%	\item sociologická,
%	\item technologická,
%	\item ekonomická,
%	\item ekologická,
%	\item politicky-právní.	
%\end{enumerate}

%\vspace*{1mm}

\textbf{Sociologická analýza}\\

Co se týče analýzy sociologických dopadů na společnost, je dle autorů třeba vyhodnotit především obecné socio-kulturní faktory v čele s demografickou analýzou. Autoři zvolili pro tuto část následující faktory: počet obyvatelstva, jeho složení podle věku, vzdělání a nezaměstnanost, inflace a růst ekonomiky.\\

Vzhledem k historickému trendu vývoji obyvatelstva (průběžný růst) od roku 1993 nelze předpok\-ládat, že by se v nejbližších letech trend výrazněji změnit. Z tohoto důvodu autoři nepovažují tento sociologický faktor za potenciálně rizikový, naopak z růstu počtu obyvatel (Viz Figure 3) a rostoucího počtu živností\footnote{Autoři zde pracují s předpokladem, že subjekty provozující živnostenské podnikání představují velkou část potenciální klientely společnosti.}\textsuperscript{,}\footnote{Vzhledem k absenci dat pro rok 2021 a probíhajíci pandemii však není vyloučeno, že pro rok 2021 má graf již sestupnou tendenci, v takovémto případě by bylo třeba daný trend vyhodnotit jako rizikový pro společnost Proxy a.s.} (viz Figure 4) se autoři domnívají, že pro společnost by tento demografický trend (zvyšování počtu obyvatel, zvyšování počtu podnikajících subjektů) mohl představ\-ovat příznivý faktor pro její vlastní ekonomický růst.\\ 

\begin{figure}[!hbtp]
	\centering
	\includegraphics[width=0.8\linewidth]{Parts/InternalResources/CHART_Population_in_time.pdf}
	\caption[Vývoj počtu obyvatel v čase]{Vývoj počtu obyvatel v čase\footfullcite{noauthor_obyvatelstvo_nodate}}
	\label{fig:Vyvoj poctu obyvatel v case}
\end{figure}

\newpage

\begin{figure}[!hbtp]
	\centering
	\includegraphics[width=0.8\linewidth]{Parts/InternalResources/CHART_Zivnosti_in_time.pdf}
	\caption[Vývoj počtu živností v čase]{Vývoj počtu živností v čase\footfullcite{noauthor_rocni_nodate}}
	\label{fig:Vyvoj poctu zivnosti v case}
\end{figure}

počet, věk, vzdělání

\newpage

\textbf{Technologická analýza}\\

Technologie v rámci analyzované společnosti nehrají zásadní smysl, k vykonávání činnosti není třeba mít nejnovější technologické vybavení a dodržovat nejnovější technologické postupy.\\

Problémy v oblasti technologického světa\footnote{Nyní je možné pozorovat na světových trzích například nedostatek výpočetních čipů.} tak na společnost nemusí mít zásadnější dopady, pokud se nebude jednat o zásadní problémy s dodávkou jakéhokoliv technologického vybavení, neboť i analyzovaná společnost potřebuje ke svojí činnosti výpočetní technologie.\\

nedostatek, špatná kvalita

\textbf{Ekonomická analýza}\\

V rámci ekonomické analýzy se autoři zaměřují především na nezaměstnanost, inflaci, růst ekonomiky, kurzové změny a platební bilanci.\\

Přidat nějaký grafy lol.\\

Poptávka, počet živnostníků, životní situace, růst ekonomika, zájem zahraničních investorů, to vše je může ohrozit. A také mnoho dalšího!\\

inflace, růst ekonomiky, zájem zahraničních investorů a kurzové změny.

\textbf{Ekologická analýza}\\

Jak již bylo řečeno v kapitole A.4 Green deal, činnost, kterou společnost vykonává, nemá zásadnější dopad na životní prostředí, to ovšem neznemená, že nemá dopad žádný, nebo naopak přispívá ke zlepšování životního přírodí (Viz kapitola V. Projektová část).\\

Ekologie jako vnější faktor však může mít na společnost zásadní dopady, především nepřímo prostřednictvím negativních dopadů na její klienty pohybující se v odvětvích, která jsou závislá na zdravém životním prostředí.\\

přímý dopad a nepřímý dopad

\textbf{Politicko právní analýza}\\

Společně s ekonomickými problémy je tento bod pro společnost největším rizikem. Česká republika má stabilní právní prostředí a relativně stabilní politickou situaci, autoři tedy nepovažují problémy spojené s náhlou změnou právního nebo politického prostředí za pravděpodobné.\\

Zásadnějším vnějším faktorem, na který si společnost musí dávat pozor jsou však novelizace právních norem. Například zák o dani z příjmu a zákon o dph byl v posledních letech novelizován X krát, přidat odkaz na ty zákony a na seznam novelizací na stránkách úřadu vlády.to ovšem nez\\

Společnost s těmito změnami musí udržovat krok, aby mohla i nadále poskytovat kvalitní služby a neublížila si na svém dobrém jméně a pověsti.\\

rapidní změna, novelizace

%\textbf{

\section*{A. Porterova analýza}
\label{sec:Porterova analyza}
\addcontentsline{toc}{subsection}{\nameref{sec:Porterova analyza}}

