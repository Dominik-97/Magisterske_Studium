\newcommand\SkenovaniVnejsihoOkoli{\texorpdfstring{\MakeUppercase{\romannumeral 2}.}{} SKENOVÁNÍ VNĚJŠÍHO OKOLÍ}
\begin{lefttextpipe}
	{\Large \SkenovaniVnejsihoOkoli}
\end{lefttextpipe}
\phantomsection
\addcontentsline{toc}{section}{\SkenovaniVnejsihoOkoli}

V rámci vnější analýzy se autoři zaměřují na 5 oblastí ze STEEP analýzy.\\

\vspace*{5mm}

Těmito oblastmi jsou:

\begin{enumerate}
	\item sociologická,
	\item technologická,
	\item ekonomická,
	\item ekologická,
	\item politicky-právní.	
\end{enumerate}

\textbf{Sociologická analýza}

Co se týče analýzy sociologických dopadů na společnost, je dle autorů třeba vyhodnotit především obecné socio-kulturní faktory v čele s demografickou analýzou. Autoři zvolili pro tuto část následující faktory: počet obyvatelstva, jeho složení podle věku, pohlaví, vzdělání a sociálního postavení, dále kulturní, náboženské a obecné změny v homogenosti/rozmanitosti obyvatelstva.\\

\begin{center}
	\begin{figure}[!hbtp]
		\includegraphics[width=14cm]{Parts/InternalResources/CHART_Population_in_time.pdf}
		\caption{Test}
		\label{fig:t}
	\end{figure}
\end{center}


\section*{A. Porterova analýza}
\label{sec:Porterova analyza}
\addcontentsline{toc}{subsection}{\nameref{sec:Porterova analyza}}