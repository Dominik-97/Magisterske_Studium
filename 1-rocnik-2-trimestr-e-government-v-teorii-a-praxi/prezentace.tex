\documentclass{beamer}

% +==============================+
% | Vydefinování maker a packagů |
% +==============================+

\usepackage[utf8]{inputenc}
\usepackage[czech]{babel}
\usepackage{graphicx}
\usepackage{tikz}
\usetikzlibrary{shapes.arrows}
\usepackage{MnSymbol,wasysym}
\usepackage{multirow}
\usepackage{xcolor}
\usepackage{filecontents}
\usepackage{etoolbox}

% +===============================================+
% | Připnutí section overview jen na levou stranu |
% +===============================================+

\makeatletter
\beamer@compresstrue
\patchcmd{\insertnavigation}{\hskip-1.875ex plus-1fill}{}{}{}
\apptocmd{\partentry}{\beamer@xpos=0\relax}{}{}
\def\sectionentry#1#2#3#4#5{%
  \ifnum#5=\c@part%
  \hbox{\def\insertsectionhead{#2}%
    \def\insertsectionheadnumber{#1}%
    \def\insertpartheadnumber{#5}%
    {%
      \usebeamerfont{section in head/foot}\usebeamercolor[fg]{section in head/foot}%
      \ifnum\c@section=#1%
        \rlap{\hyperlink{Navigation#3}{{\usebeamertemplate{section in head/foot}}}}%
      \fi}%
  }%
  \fi\ignorespaces}
\makeatother

% +===============================+
% | Makro na zobrazení CEVRO Loga |
% +===============================+

\newcommand{\cevroLogo}{
	\begin{tikzpicture}[remember picture,overlay]  
		\node [xshift=-1.6cm,yshift=-0.5cm] at (current page.north east)
		{\includegraphics[width=3cm,height=2cm,keepaspectratio]{images/newcevro}};
	\end{tikzpicture}
}

% +=========================================+
% | Makro na používání hyperlinků v listech |
% +=========================================+

\newcounter{totenum}
\setcounter{totenum}{0}
\let\Oldenumerate\enumerate
\def\enumerate{\refstepcounter{totenum}\Oldenumerate}
\newcommand{\myitem}[1]{\refstepcounter{enumi}\item[\hyperlink{#1}{\arabic{totenum}.\arabic{enumi}}]}

% +===========================================================+
% | Změna velikosti footnotů a odstranění navigačních symbolů |
% +===========================================================+

\renewcommand{\footnotesize}{\tiny}
\beamertemplatenavigationsymbolsempty

% +=========================+
% | Vydefinování tikz stylů |
% +=========================+

\tikzset{
    myarrow/.style={
        draw,
        fill=green,
        single arrow,
        minimum height=3.5ex,
        single arrow head extend=1ex
    }
}

\tikzset{
    myarrowred/.style={
        draw,
        fill=red,
        single arrow,
        minimum height=3.5ex,
        single arrow head extend=1ex
    }
}

% +=================================+
% | Základní informace o prezentaci |
% +=================================+

\title{e-Government - Datové schránky: přínosy a negativa}
\author[My name]{Dominik Bálint}
\institute{CEVRO Institut}
\date{\today}

% +=============+
% | První sekce |
% +=============+

\section{Datové schránky}

% +===============================================+
% | Použitý theme v prezentaci k bakalářské práci |
% +===============================================+

\usetheme{Singapore}

% +=================================+
% | Vydefinování použité literatury |
% +=================================+

\begin{filecontents}{\jobname.bib}

@TECHREPORT{RePEc:cwl:cwldpp:159,
    title = {Commercial Banks as Creators of 'Money'},
    author = {Tobin, James},
    year = {1963},
    institution = {Cowles Foundation for Research in Economics, Yale University},
    type = {Cowles Foundation Discussion Papers},
    number = {159},
    url = {http://EconPapers.repec.org/RePEc:cwl:cwldpp:159}
}

@TECHREPORT{RePEc:cwl:cwldpp:160,
    title = {Commercial Banks as Creators of 'Money'},
    author = {Tobin, James},
    year = {1963},
    institution = {Cowles Foundation for Research in Economics, Yale University},
    type = {Cowles Foundation Discussion Papers},
    number = {159},
    url = {http://EconPapers.repec.org/RePEc:cwl:cwldpp:159}
}
\end{filecontents}

\usepackage[style=verbose,backend=biber]{biblatex}
\addbibresource{\jobname.bib}

\begin{document}

% +===============+
% | Titulní slide |
% +===============+

\maketitle

% +====================+
% | Začátek prezentace |
% +====================+

\begin{frame}
	\frametitle{Obsah}
	\cevroLogo

\begin{enumerate}
	\myitem{PREDSTAVENI} Představení datových schránek
	\myitem{HISTORIE} Historie datových schránek
	\myitem{ZALOZENI} Založení datové schránky
	\myitem{PRINOSY} Přínosy datové schránky
	\myitem{NEGATIVA} Negativa datové schránky
\end{enumerate}
	
\end{frame}

% +=============+
% | Druhý slide |
% +=============+

\begin{frame}
	\frametitle{Představení datových schránek}
	\cevroLogo
	\label{PREDSTAVENI}

Ahoj.

\end{frame}

% +=============+
% | Třetí slide |
% +=============+

\begin{frame}
	\frametitle{Historie datových schránek}
	\cevroLogo
	\label{HISTORIE}

Ahoj.

\end{frame}

% +==============+
% | Čtvrtý slide |
% +==============+


\begin{frame}[t]
	\frametitle{Založení datové schránky}
	\cevroLogo
	\label{ZALOZENI}
	
Zřízení datové schránky je:

\begin{enumerate}
	\item povinné ze zákona, nebo \footcite{RePEc:cwl:cwldpp:159}
	\item na žádost. \footcite{RePEc:cwl:cwldpp:160}
\end{enumerate}

\vspace*{5mm}

\underline{\large Povinné zřízení datové schránky} \newline

Ahoj. \newline

\underline{\large Zřízení datové schránky na žádost} \newline

Ahoj.
	
\end{frame}

% +==============+
% | Pátý slide   |
% +==============+


\begin{frame}
	\frametitle{Přínosy datové schránky}
	\cevroLogo
	\label{PRINOSY}
	
Ahoj

Ahoj
	
\end{frame}

% +==============+
% | Šestý slide  |
% +==============+


\begin{frame}[t]
	\frametitle{Negativa datové schránky}
	\cevroLogo
	\label{NEGATIVA}
	
Ahoj \newline

\vspace*{-1cm}
\begin{table}[]
\begin{tabular}{|l|}
\hline
Otázka 2 Oponent práce                                                                                                                                                                                                                      \\ \hline
\begin{tabular}[c]{@{}l@{}}Autor by mohl dále shrnout některé dílčí odlišnosti québecké\\ a české úpravy svěřenských fondů a tím učinit závěr v dané otázce\\ na str. 20 až 23.\end{tabular} \\ \hline
\end{tabular}
\end{table}
	
\end{frame}

% +===============+
% | Sekce zdroje  |
% +===============+

\section{Zdroje}

% +==============+
% | Sedmý slide  |
% +==============+

 \begin{frame}
 	\frametitle{Zdroje}
	\cevroLogo
    	\printbibliography[heading=none]
 \end{frame}
	
\end{document}